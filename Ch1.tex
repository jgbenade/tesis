\chapter{Introduction}
% put these two lines after every \chapter{} command
\vspace{-2em}
\minitoc
\startarabicpagenumbering % must be just after the first \chapter{} command

Chapter 1 will introduce the topic of \lat s in a gripping way so that the readers interest is piqued enough to continue reading. It will formally define the problem tackled in this thesis, set clear objectives for the thesis and delimit the scope of the research in a reasonable and sensible way. Finally, this chapter will also give an overview of what the reader may expect from the rest of the thesis and briefly outline the content of the succeeding chapters.

\section{Historical background}

\section{Problem statement}

\section{Scope and objectives}
Objectives:
\begin{itemize}
\item Survey the literature on \lat  s and distributed computing projects
\item Study the state-of-the-art techniques used for enumerating $k$-MOLS
\item Design a fast algorithm for enumerating main classes of $k$-MOLS
\item Implement said algorithm and verify results by comparing it to published findings
\item Design and launch a distributed computing project
\item Obtain (novel) results from the distributed enumeration
\item Contribute towards answering the celebrated  existence question of a 3-MOLS of order 10 by estimating the effectiveness of a distributed enumeration approach
\end{itemize}

Scope:
Combinatorial designs other that \lat are considered to be beyond the scope of this thesis, except in cases where the use of these designs may contribute towards a clearer understanding of some topic.
The thesis will focus on enumerating main classes of $k$-MOLS, any other equivalence classes are considered to be beyond the scope.


\section{Thesis organisation}