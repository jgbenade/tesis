%%%%%%%%%%%%%%%%%%%%%%%%%%%%%%%%%%%%%%%%%%%%%%%%
%
% some tips on creating nice tables
%
%%%%%%%%%%%%%%%%%%%%%%%%%%%%%%%%%%%%%%%%%%%%%%%%
\chapter{Tables}

% TABLE SHADING
%===============
%
% uncomment to change the default shades (gray!45 and gray!25) for the tables in your thesis:
%    \colorlet{tableheadcolor}{gray!45}  % headers
%    \colorlet{tablerowcolor}{gray!25}   % normal rows
%
% \headcol                 - for shading the header of your table
% \rowcol                  - shades an entire row
% \rowcolor{color}         - for custum coloring of a row
%
% \cellcolrow              - shades a single entry in the table the shade of a row
% \cellcolhead             - shades a single entry in the table the shade of the header
% \cellcolor{color}        - for custom colouring of a single entry in the table
%
% \rowcolors{startrow}{oddrowcolor}{evenrowcolor} - for automatic shading of rows, put in table environment
%   examples:
%     \rowcolors{3}{gray!25}{} - shades 3rd row and every second row after that
%     \rowcolors{2}{}{gray!25} - shades 2rd row and every second row after that




% TABLE LINES
%=============
%
% \toprule     - the top-most line of a table, does not work with shading
% \hline       - normal line, does not work with shading
% \midline     - normal line, does not work with shading
% \bottomrule  - a line for the bottom of the table, does not work with shading
%
% \topline     - the top-most line of a table if the header is shaded
%
% \midline     - the line between the headings and the table body
% \midlinecbw  - a line for when the previous row is rowcolor and the next line is white
% \midlinecw   - a line with no black, to further separate a rowcolor row and a white row
% \midlinewbc  - a line for when the upper row is white and the next line is rowcolor
% \midlinewc   - a line with no black, to further separate a white row and a rowcolor row
%
% \bottomline  - a line for the bottom of the table, when the last row is white
% \bottomlinec - a line for the bottom of the table, when the last row is rowcolor




% EXAMPLE
%========

\begin{table}[h!tb]

\centering

\rowcolors{3}{gray!25}{}

\begin{tabular}{ccc}\topline
\headcol       & & \\ \midline
\hspace{5cm} \ & & \\
               & & \\ 
               & & \\ 
               & & \\ 
               & & \\ 
               & & \\ \bottomlinec
\end{tabular}

\caption[Example 1 as in list of tables]{Example 1}
\label{ex1}

\end{table}




% EXAMPLE
%=========
%
% example using multicolumn, multirow and the sideways environment

\begin{table}[h!tb]

\centering

\begin{tabular}{cc|rrrrrr}\hline

&&\multicolumn{6}{c}{this goes across 6 columns}\\

 && col a & col b & col c & col d & col e & col f \\ \hline \hline

\multirow{6}{*}{
%
\begin{sideways}
this is sideways,
\end{sideways}
%
\begin{sideways}
and goes across
\end{sideways}
%
\begin{sideways}
six rows
\end{sideways}
%
}

& row 1 \\
& row 2 \\
& row 3 \\
& row 4 \\
& row 5 \\
& row 6 \\ \hline

\end{tabular}

\caption[Example 2 as in list of tables]{Example 2}
\label{ex2}

\end{table}


