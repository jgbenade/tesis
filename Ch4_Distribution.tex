\chapter{The design of a distributed computing project}
% put these two lines after every \chapter{} command
\vspace{-2em}
\minitoc

Chapter 4 introduces the concept of distributed computing by  giving its history and a summary of some well-known projects, together with their results and the frameworks they were built on. The different distribution frameworks that are available today will then be surveyed.  The working of Berkeley Open Infrastructure for Network Computing, in particular, will be examined in detail, as well as all the requirements for a BOINC project.  The chapter will conclude in specifying the design choices neccessary to create a  BOINC project for the enumeration of main classes of \lat s, for example how to break the search tree up into workunits, how to assign workunits to a user based on system specifications etc.

\section{A brief history of distributed computing}

\section{Using Berkeley Open Infrastructure for Network Computing}

\subsection{Basic workflow}

\subsection{The components of a distributed computing project}

\section{Designing a BOINC project for the enumeration of MOLS}






\section{Chapter summary}