\chapter{Deployment of the distributed enumeration project}
% put these two lines after every \chapter{} command
\vspace{-2em}
\minitoc

Chapter 5 will contain details pertaining to the method of deploying a small-scale distributed search based on the design set out in Chapter 4 as a proof of concept.  It will include how awareness of the project was raised in the scientific community, the  challenges involved in launching the project as well as the lessons learnt for the possible future deployment of larger scale projects.  The chapter will conclude with the results of the distributed computing project designed in Chapter 4, most likely the enumeration of main classes of MOLS of order 8 or 9.  It will also give a summary of the distributed enumeration in terms of number of volunteers, locality, how work was done by each of the volunteers etc. 



\section{Creating a volunteer computing project for the enumeration of 3-MOLS of order 8}
\subsection{Server architecture}
virtual machine
\subsection{Grid-enabling the exhaustive enumeration algorithm}
starts, checkpoints, 'in', 'out'
\subsection{Deamons}
\subsection{Preliminary enumeration results}


\section{Generalizing the project for the enumeration of $k$-MOLS of order $n$}
\subsection{Limiting workunit sizes}
number of calls, serial, not huge workunit, format of output files.
\subsection{Dynamically splitting   workunits}
limits
\subsection{Comparing enumeration results}
Compare total number of workunits, length of longest, time from start to completion etc

\section{Public launch}

\section{Chapter summary}